The AeroTech X9 is engineered to perform reliably and effectively across diverse and challenging operational environments, from urban landscapes to remote terrains, and in varying weather conditions. Many considerations have been taken in the design of the X9 to accommodate battery performance and electrical component selection. Aerotech drones are also rigorously tested in simulation and real-world conditions.  \\
\begin{itemize}
    \item Environmental Factors:
    \begin{itemize}
        \item Weather Conditions
        \begin{itemize}
            \item Temperature \\
            The X9 is able to operate in harsh climates like those found in high altitudes, the coldest seasons, and the unforgiving climates in middle eastern deserts. The range in temperature for the X9 is between -20 degrees to +50 degrees Celsius(-4 to 122 Fahrenheit)
            \item Precipitation
            \\
            The propellors of the X9 are optomized to maintain lift in moderate precipitation, and all external connectors are waterproof in their design. Each of the electrical components are stored in sealed enclosures with water resistant coatings. 
            \item Wind
            \\
            
            \item Visibility
        \end{itemize}
        \item Terrain and Geography
        \begin{itemize}
            \item Urban Environments
            \item Rural/Remote Areas
            \item Maritime Environments
        \end{itemize}
    \end{itemize}
    \item Operational 
    Considerations:
    \begin{itemize}
        \item Regulatory Compliance
        \item Logistics and Support
        \item Safety and Risk Management
    \end{itemize}
    \item Case Studies:
    \begin{itemize}
        \item Urban Infrastructure Inspections
        \item Rural Search and Rescue
        \item Maritime Surveillance    
    \end{itemize}
\end{itemize}
