\documentclass{article}

\usepackage{hyperref}
\usepackage{titlesec}
\usepackage{multirow}
\usepackage[table]{xcolor}
\usepackage{colortbl}
\usepackage{array}
\usepackage{makecell}
\usepackage{soul} % add \st{} for strikthrough
\usepackage[letterpaper, portrait, margin=1in]{geometry}
\usepackage{cite}
\usepackage{indentfirst}
% \setlength\parindent{0pt}

% report title
\title{Assignment 1 - System Categroization \\ \textit{Security Categorization}\\ CSE 4380}

\author{Group Thorin}

% date of assignment submission
\date{February 28th, 2025}

% Colors for Security Categorizations
\definecolor{navyblue}{RGB}{0,0,128}
\definecolor{lightgray}{RGB}{211,211,211}

\begin{document}

\maketitle
\begin{center}
\begin{tabular}{l r}

% team members
Members: 	& Obadah Al-Smadi\\
			& Betim Hodza\\
			& Elliot Mai\\
			& Benjamin Niccum\\
			& Nicholas Pratt\\
Instructor: & Trevor Bakker\end{tabular}
\end{center}

\newpage

\setcounter{tocdepth}{2}
\tableofcontents
\newpage

% List of figures and tables (optional)
\listoffigures
\listoftables
\newpage

\section{Executive Summary}
\par This whitepaper evaluates the security classification of the 
AeroTech X9 drone system using established federal guidelines. 
Originally designed for commercial applications, the X9 is now 
being upgraded to meet military-grade specifications for defense 
use. This transition introduces heightened security requirements 
due to its enhanced surveillance, intelligence-gathering, and 
operational capabilities. By analyzing the confidentiality, 
integrity, and availability of its data, this document assigns 
a security classification to guide risk management and compliance 
measures.

\section{Introduction}
\par The upgrade of the AeroTech X9 drone from commercial to military 
use introduces new and enhanced information systems that require proper 
categorization and vulnerability analysis. Furthermore, existing systems 
from the commercial iteration must be reevaluated, as they could interact 
with new features in unexpected ways, potentially creating security risks. 

\section{Purpose}
\par The purpose of this whitepaper is to evaluate and assign a security 
classification to the AeroTech X9 drone system based on its transition 
from commercial to military-grade use. This classification will guide 
risk management strategies and compliance with relevant security 
frameworks, ensuring secure operations in defense environments.

\section{Security Categorization}

\subsection{Information Types}

\begin{itemize}
    \item GNSS / GPS: Our GNSS device communicates using Chips-Message Robust Authentication (Chimera) and provides secure communication between the drone and the operator. It will switch to one of the 3 other modules if it fails or gets jammed.
    \item LTE / 4G: Our LTE/4G communications uses TLS 1.2 and provides secure communication between the drone and the operator. It will independtly switch when jamming occurs or when loss of function occurs in one of the 3 modules of communication.
    \item SatNav:  Our Satnav allows for Beyond-Line-of-Sight (BLOS) Operations, and will be encrypted with AES-256-GCM. it will also switch if one of the 3 modules fail.
    \item (IMU) Intertial Measurement Unit: The IMU is used to measure the drones position, it contains critical data to keep the drone flying correctly. If tampered it could result in lost control of the system and potential loss of life.
    \item Telementry Module: Used for real-time monitoring and secure data transmission between the drone and ground station. It utilizes AES-256 encryption with secure key exchange (Diffie-Hellman or ECC) to prevent interception and maintain integrity.
    \item Flight Control board: Our Flight Control Board has Automatic actions on waypoints: suitable for cargo drop or camera shots, Transponder ADS-B IN for UTM (Unmanned Traffic Management),  and Flare and parachute activation for target drones. This module helps ensure availability of the drone and protects it from worst case scenarios. The confidentiality is important to consider for this device as it has data flowing into it that could be confidential and vital to its integrity (drone drop points and return points).
    \item High-Res Camera: Our High-Resolution camera is equipped with advanced security features to ensure the integrity and confidentality of captured data. The camera stores data using AES-256-CBC on an full encrypted drive.
    \item Thermal Imaging Camera: Thermal imaging sensor for night operations and heat detection. If sensor data intercepted it could reveal mission critical information.
    \item AeroTech Flight Software: AeroTech's inhouse flight software is designed with security implementation first, making sure that the integrity of flight software is the upmost importance. As well as keeping confidential information secure and encrypted in memory. 
    \item Encrypted Data (flight, image, comms data):  The Drone process image data and flight data, as well as has secure encrypted communication standards in place when in transit TLS 1.2 (telementry module). It also keeps data encrypted at rest as well when storing image data (AES-256-CBC).
    \item LIDAR sensor: Used for topographical mapping and obstacle detection. If sensor data intercepted it could reveal mission critical information.
    \item CPU: Used for executing all programs and commands on the drone.
    \item RAM: Used for hold all process currently being executed.
    \item SSD: Used as a non-volatile alternative to RAM for storing Operating System data as well as encryption key and authetication certificates..
\end{itemize}

\subsection{Impact Levels (Confidentiality, Integrity, Availability)}

\begin{center}
    \rowcolors{2}{lightgray}{white}
    \begin{tabular}{|p{4cm}|p{3.5cm}|p{3.5cm}|p{3.5cm}|}
    \hline
    \rowcolor{navyblue!80}
    \color{white}\textbf{Information Types} & 
    \color{white}\textbf{Confidentiality} & 
    \color{white}\textbf{Integrity} & 
    \color{white}\textbf{Availability} \\ \hline
    
    % GNSS/GPS Module
    \makecell{GNSS/GPS Module} & 
    \makecell[l]{L\\ \scriptsize Disclosure to the information sent between the GNSS \\ \scriptsize module will lead to minimal \\ \scriptsize impact in the system. Most \\ \scriptsize of the data sent between \\ \scriptsize the module is GNSS \\ \scriptsize coordinates they aren't highly \\ \scriptsize sensitive. (\cite{nistsp80060v1r1} 3.2)} & 
    \makecell[l]{L\\ \scriptsize Manipulation of the module \\ \scriptsize could lead to minimal impact \\ \scriptsize in integrety, as we can switch \\ \scriptsize to another module instead of \\ \scriptsize using GNSS. (\cite{nistsp80060v1r1} 4.2.2.2)} & 
    \makecell[l]{L\\ \scriptsize Due to loss of availability \\ \scriptsize there will be minimal impact \\ \scriptsize to the systems capability, \\ \scriptsize since we can switch to other \\ \scriptsize methods of communication. \\ \scriptsize (\cite{nistsp80060v1r1} 4.2.2.3)} \\ \hline
    
    % LTE/4G
    \makecell{LTE/4G} & 
    \makecell[l]{M\\ \scriptsize Disclosure to the information \\ \scriptsize sent between the LTE/4G \\ \scriptsize module and the controller \\ \scriptsize could lead to Serious loss of \\ \scriptsize human life, as an adversary \\ \scriptsize can contain extra sensitive \\ \scriptsize data as it's not only being \\ \scriptsize used for control of the drone, \\ \scriptsize but in link with the \\ \scriptsize telementry module too.  \\ \scriptsize (\cite{nistsp80060v1r1} 4.2.2.1)} & 
    \makecell[l]{L\\ \scriptsize Manipulation of the module \\ \scriptsize could lead to minimal impact \\ \scriptsize in integrety, as we can switch \\ \scriptsize to another module instead of \\ \scriptsize LTE / 4G if there's \\ \scriptsize interference. (\cite{nistsp80060v1r1} 4.2.2.2)} & 
    \makecell[l]{L\\ \scriptsize Due to loss of availability \\ \scriptsize there will be minimal impact \\ \scriptsize to the systems capability, \\ \scriptsize since we can switch to other \\ \scriptsize methods of communication. \\ \scriptsize (\cite{nistsp80060v1r1} 4.2.2.3)} \\ \hline
    
    % SATNAV
    \makecell{SATNAV} & 
    \makecell[l]{L\\ \scriptsize TDisclosure to the \\ \scriptsize information sent between the \\ \scriptsize SatNav module will lead to \\ \scriptsize minimal impact in the system, \\ \scriptsize as most of the data is \\ \scriptsize satalite related navigation. \\ \scriptsize (\cite{nistsp80060v1r1} 3.2)} & 
    \makecell[l]{L\\ \scriptsize Manipulation of the module \\ \scriptsize could lead to minimal impact \\ \scriptsize in integrety, it does limit the \\ \scriptsize range of the drone as there's \\ \scriptsize no over the horizion \\ \scriptsize communication that could be \\ \scriptsize done proficiently. \\ \scriptsize (\cite{nistsp80060v1r1} 4.2.2.2)} & 
    \makecell[l]{L\\ \scriptsize  Due to loss of availability \\ \scriptsize there will be minimal impact \\ \scriptsize to the systems capability, \\ \scriptsize since we can switch to other \\ \scriptsize methods of communication \\ \scriptsize (\cite{nistsp80060v1r1} 4.2.2.3)} \\ \hline
    
    % IMU
    \makecell{IMU} & 
    \makecell[l]{L\\ \scriptsize If IMU data is disclosed \\ \scriptsize there's minimal impact of \\ \scriptsize confidentiality, the data is \\ \scriptsize mostly used for the drones \\ \scriptsize orientation for autoflight, or \\ \scriptsize normal flight procedures. \\ \scriptsize (\cite{nistsp80060v1r1} 4.2.2.1)} & 
    \makecell[l]{H\\ \scriptsize Severe impact may occur  \\ \scriptsize that could include a loss of \\ \scriptsize life if the IMU is manipulated. \\ \scriptsize  It's critical that we ensure \\ \scriptsize accurate measurements of \\ \scriptsize motion and orientation or the \\ \scriptsize drone will be inorperable.  \\ \scriptsize (\cite{nistsp80060v1r1} 4.2.2.2)} & 
    \makecell[l]{H\\ \scriptsize Without the IMU the drone \\ \scriptsize can't autocorrect itself, \\ \scriptsize which can cause it to crash \\ \scriptsize and not fly correctly even \\ \scriptsize with autopilot. (\cite{nistsp80060v1r1} 4.4.2.4)} \\ \hline
    
    % Telementry Module
    \makecell{Telementry Module} & 
    \makecell[l]{M\\ \scriptsize Disclosure of the information \\ \scriptsize transmitted between the \\ \scriptsize Module could lead to  \\ \scriptsize significant unauthorized  \\ \scriptsize access to sensitive data of the \\ \scriptsize operations being taken in \\ \scriptsize place. This could comprimise \\ \scriptsize the security of comms and \\ \scriptsize allow adversaries to intercept \\ \scriptsize transmissions and know the \\ \scriptsize purpose of a mission. \\ \scriptsize (\cite{nistsp80060v1r1} 4.2.2.1)} & 
    \makecell[l]{L\\ \scriptsize Changing the data \\ \scriptsize transmitted in the module \\ \scriptsize could lead to errors and \\ \scriptsize inconsistencies between \\ \scriptsize communication. This isn't \\ \scriptsize our only option to \\ \scriptsize communicate so it'll \\ \scriptsize minimally impact the \\ \scriptsize mission.  (\cite{nistsp80060v1r1} 4.2.1 Table 7.)} & 
    \makecell[l]{L\\ \scriptsize Loss of availability would \\ \scriptsize have minimal impact of \\ \scriptsize overall system capabilities. \\ \scriptsize Alternative Communication \\ \scriptsize methods can be employed to \\ \scriptsize maintain operation. \\ \scriptsize  (\cite{nistsp80060v1r1} 4.2.1 Table 7, 4.2.2.3)} \\ \hline
    
    
    
    
    \end{tabular}
    
    % Table 2 for more data
    
    \begin{tabular}{|p{4cm}|p{3.5cm}|p{3.5cm}|p{3.5cm}|}
    \hline
    \rowcolor{navyblue!80}
    \color{white}\textbf{Information Types} & 
    \color{white}\textbf{Confidentiality} & 
    \color{white}\textbf{Integrity} & 
    \color{white}\textbf{Availability} \\ \hline
    
    % Flight Control Board
    \makecell{Flight Control Board} & 
    \makecell[l]{L\\ \scriptsize Disclosure of information \\ \scriptsize processed on the board could \\ \scriptsize lead to minor operational \\ \scriptsize disruptions. This could \\ \scriptsize compromise the security of \\ \scriptsize the drones system allowing \\ \scriptsize for potential exploits to be \\ \scriptsize done. (\cite{nistsp80060v1r1} 3.2)} & 
    \makecell[l]{H\\ \scriptsize Severe impact will occur if \\ \scriptsize the integrity of the FCB is \\ \scriptsize comprimised, as it's used for \\ \scriptsize not only autopilot but helps \\ \scriptsize assit in manual flight for an \\ \scriptsize operator. \\ \scriptsize (\cite{nistsp80060v1r1} 4.2.2.2 Table 7)} & 
    \makecell[l]{H\\ \scriptsize Severe impact will occur if \\ \scriptsize the Availability of the FCB \\ \scriptsize is lost, similarly to Integrity \\ \scriptsize it will loose autopilot but \\ \scriptsize normal controls for flying \\ \scriptsize  the drone manually. \\ \scriptsize  (\cite{nistsp80060v1r1} 4.4.2.4)} \\ \hline

    % High-Res Camera
    \makecell{High-Res Camera} & 
    \makecell[l]{L\\ \scriptsize The images themselves can \\ \scriptsize contain confidential \\ \scriptsize information, leading to \\ \scriptsize potential mission operations \\ \scriptsize distruptions. But in terms of \\ \scriptsize comprimises to operations \\ \scriptsize this will be minimal. \\ \scriptsize  (\cite{nistsp80030r1}4.2.2.1.)} & 
    \makecell[l]{L\\ \scriptsize Minimal impact of the system \\ \scriptsize will occur for tampering \\ \scriptsize image data for the camera. \\ \scriptsize We have other camera's to \\ \scriptsize switch to and from, and it \\ \scriptsize won't disrupt operations \\ \scriptsize much. (\cite{nistsp80060v1r1} 4.2.2.2)} & 
    \makecell[l]{L\\ \scriptsize Minimal impact could occur \\ \scriptsize for the loss of availability of \\ \scriptsize the camera, it will impact \\ \scriptsize mission capability as if \\ \scriptsize we switch to thermal \\ \scriptsize imaging camera, it won't \\ \scriptsize be the best way to operate \\ \scriptsize the drone (\cite{nistsp80060v1r1}4.4.2.3)} \\ \hline
    
    % Thermal Camera
    \makecell{Thermal Camera} & 
    \makecell[l]{L\\ \scriptsize The images themselves can \\\scriptsize contain confidential  \\\scriptsize information, but the image clarity of  \\\scriptsize  themeral cameras won't lead to major mission  \\\scriptsize  distruptions. But in terms of comprimises  \\\scriptsize to operations, this will be minimal.\\ \scriptsize (\cite{nistsp80060v1r1} 3.2)} & 
    \makecell[l]{L\\ \scriptsize Minimal impact of the \\ \scriptsize system will occur for \\ \scriptsize tampering image data for the \\ \scriptsize thermal camera. It can be \\ \scriptsize an inconvience but as long \\ \scriptsize as we have redundant \\ \scriptsize systems it won't disrupt \\ \scriptsize operations much.\\ \scriptsize (\cite{nistsp80060v1r1} 4.2.2.2)} & 
    \makecell[l]{L\\ \scriptsize Minimal impact could occur \\ \scriptsize for the loss of availability \\ \scriptsize of the thermal camera, it \\ \scriptsize will impact mission \\ \scriptsize capabilities as we could \\ \scriptsize get crucial information \\ \scriptsize from thermal cameras, but \\ \scriptsize it won't be huge disruptions. \\ \scriptsize (\cite{nistsp80060v1r1} 4.4.2.4)} \\ \hline
    
    % CPU
    \makecell{CPU} & 
    \makecell[l]{H\\ \scriptsize With unauthorized access \\ \scriptsize an attacker could \\ \scriptsize extract cryptographic keys \\ \scriptsize from the CPU allowing them \\ \scriptsize to decrypt messages on all \\ \scriptsize other parts of the system or \\ \scriptsize perform a memory dump to \\ \scriptsize extract flight plans and \\ \scriptsize other mission critical data.\\ \scriptsize (\cite{nistsp80060v1r1} 4.2.2.1)} & 
    \makecell[l]{H\\ \scriptsize With unauthorized access an \\ \scriptsize attacker could cause \\ \scriptsize catastrophic damage to the \\ \scriptsize system. They could overload \\ \scriptsize the CPU and causing delays \\ \scriptsize in the processing of sensor \\ \scriptsize data or cause a kernel panic, \\ \scriptsize cutting off connection to \\ \scriptsize ground control, and causing \\ \scriptsize the drone to crash. \\ \scriptsize (\cite{nistsp80060v1r1} 4.4.2.4)} & 
    \makecell[l]{H\\ \scriptsize if the CPU becomes \\ \scriptsize unavailble, severe \\ \scriptsize consequences in mission \\ \scriptsize capabilities will occur, \\ \scriptsize causing our drone to be \\ \scriptsize inopperable. \\ \scriptsize (\cite{nistsp80060v1r1} 4.4.2.4 and 4.2.2.3)} \\ \hline
    % RAM
    \makecell{RAM} & 
    \makecell[l]{M\\ \scriptsize Volatile memory could \\ \scriptsize disclose severe mission \\ \scriptsize critical data, although the \\ \scriptsize difficulty to exploiting this \\ \scriptsize is harder then some of the \\ \scriptsize others, it's a possibility.\\ \scriptsize (\cite{nistsp80060v1r1} 4.2.2.1)} & 
    \makecell[l]{M\\ \scriptsize Serious impact could lead to \\ \scriptsize interruptions to the drone's \\ \scriptsize operation, but this is also \\ \scriptsize unlikly to occur. Changing \\ \scriptsize the RAM data would require \\ \scriptsize a lot of work but could mean \\ \scriptsize that the system is already \\ \scriptsize comprimised from some other \\ \scriptsize reason.\\ \scriptsize (\cite{nistsp80060v1r1} 4.4.2.2)} & 
    \makecell[l]{H\\ \scriptsize If volatile memory is \\ \scriptsize unavailbile critical mission \\ \scriptsize capabilities will not be \\ \scriptsize available and will lead to \\ \scriptsize catastrophic damage.\\ \scriptsize (\cite{nistsp80060v1r1} 4.4.2.4,  4.2.2.3.)} \\ \hline
    
\end{tabular}
\end{center}

\begin{center}
\begin{tabular}{|p{4cm}|p{3.5cm}|p{3.5cm}|p{3.5cm}|}
    \hline
    \rowcolor{navyblue!80}
    \color{white}\textbf{Information Types} & 
    \color{white}\textbf{Confidentiality} & 
    \color{white}\textbf{Integrity} & 
    \color{white}\textbf{Availability} \\ \hline

    % SSD
    \makecell{SSD} & 
    \makecell[l]{H\\ \scriptsize With unauthorized access, \\ \scriptsize an attacker could steal data \\ \scriptsize from the SSD, leading to \\ \scriptsize the exposure of classified \\ \scriptsize mission system information \\ \scriptsize and mission-critical data.\\\scriptsize (\cite{nistsp80060v1r1} 4.2.2.1)} & 
    \makecell[l]{H\\ \scriptsize With unauthorized access, \\ \scriptsize an attacker could modify any \\ \scriptsize data stored on the SSD such \\ \scriptsize flight plans or insert new \\ \scriptsize data such as malware. \\ \scriptsize\\ \scriptsize (\cite{nistsp80060v1r1} 4.4.2.4)} & 
    \makecell[l]{M\\ \scriptsize  With unauthorized access, \\ \scriptsize an attacker could corrupt or \\ \scriptsize wipe the data stored on \\ \scriptsize the SSD, rendering it \\ \scriptsize inaccessible.\\\scriptsize (\cite{nistsp80060v1r1} 4.4.2.3)} \\\hline
    
    % AeroTech Flight Software
    \makecell{AeroTech Flight\\ Software} & 
    \makecell[l]{M\\ \scriptsize The flight software includes \\ \scriptsize proprietary and sensitive \\ \scriptsize information within, this can \\ \scriptsize lead to serious breaches in \\ \scriptsize confidentiality if our \\ \scriptsize adversaries copy or use our \\ \scriptsize techniques.\\ \scriptsize (\cite{nistsp80060v1r1} 4.2.2.1)} & 
    \makecell[l]{H\\ \scriptsize Severe damage and loss of \\ \scriptsize human life could occur if the \\ \scriptsize integrity is comprimised, as \\ \scriptsize the flight controller is \\ \scriptsize important to keep the drone \\ \scriptsize operating properly, if it's \\ \scriptsize manipulated in some sort of \\ \scriptsize way it could severly disrupt \\ \scriptsize operations.\scriptsize (\cite{nistsp80060v1r1} 4.4.2.2)} & 
    \makecell[l]{H\\ \scriptsize Loss of availability will mean \\ \scriptsize that the drone can no longer \\ \scriptsize operate properly. Severe \\ \scriptsize damage could happen if \\ \scriptsize availability is lost during \\ \scriptsize flight with the drone, as it's \\ \scriptsize a mission critical device. \\\scriptsize (\cite{nistsp80060v1r1} 4.4.2.4)} \\ \hline
    
    % Encrypted Data
    \makecell{Encrypted Data} & 
    \makecell[l]{M\\ \scriptsize if the encryption keys for any \\ \scriptsize of the data get exposed then \\ \scriptsize serious mission data gets \\ \scriptsize exposed and can cause \\ \scriptsize damage.\\ \scriptsize (\cite{nistsp80060v1r1} 4.2.2.1)} & 
    \makecell[l]{M\\ \scriptsize Tampering of encrypted data \\ \scriptsize can lead to significant \\ \scriptsize distruptions to operations, as \\ \scriptsize it could be mission critical \\ \scriptsize to keep it secure.\\ \scriptsize (\cite{nistsp80060v1r1} 4.2.2.2)} & 
    \makecell[l]{L\\ \scriptsize Loss of availability of this \\ \scriptsize data  doesn't affect mission \\ \scriptsize critical resources or affects \\ \scriptsize the drone much.\\ \scriptsize (\cite{nistsp80060v1r1} 4.2.2.3)} \\ \hline
            

\end{tabular}
\end{center}
    

% Overall Categorization
\subsection{Overall Categorization}
\begin{center}
\begin{tabular}{|p{4cm}|p{3.5cm}|p{3.5cm}|p{3.5cm}|}
\hline
\rowcolor{navyblue!80}
% header setup
\color{white}\textbf{Overall Categorization} & 
\color{white}\textbf{Confidentiality} & 
\color{white}\textbf{Integrity} & 
\color{white}\textbf{Availability} \\ \hline

% Overall categorization info
\makecell{High impact system: \\ NIST-800-60: 4.4.3} & 
\makecell[l]{H} & 
\makecell[l]{H} & 
\makecell[l]{H} \\ \hline
\end{tabular}
\end{center}

\subsection{Reference Standards}
\begin{itemize}
    \item FIPS 199: Standards for Security Categorization of Federal Information and Information Systems
    \item NIST SP 800-60: Guide for Mapping Types of Information and Information Systems to Security Categories
\end{itemize}


\section{Risk Management \& Compliance}
\par Alignment with Risk Management Framework (RMF) per NIST SP 800-37. 
Categorization informs security control selection (NIST SP 800-53). 
Ongoing assessment and mitigation per FIPS 200 minimum security controls.

\section{Conclusion}
\par Example citation, remove later. (4.2.2.1 \cite{fips199})
\par (add conclusion)


% References
\bibliographystyle{plain}
\bibliography{references/refs}{}

\end{document}