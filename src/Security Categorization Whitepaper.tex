\documentclass{article}

\usepackage{hyperref}
\usepackage{titlesec}
\usepackage{multirow}
\usepackage[table]{xcolor}
\usepackage{colortbl}
\usepackage{array}
\usepackage{makecell}
\usepackage[letterpaper, portrait, margin=1in]{geometry}
\setlength\parindent{0pt}

\renewcommand{\theadfont}{\normalsize\bfseries}

% report title
\title{Assignment 1 - System Categroization \\ \textit{Security Categorization}\\ CSE 4380}

\author{Group Thorin}

% date of assignment submission
\date{February 28th, 2025}

% Colors for Security Categorizations
\definecolor{navyblue}{RGB}{0,0,128}
\definecolor{lightgray}{RGB}{211,211,211}

\begin{document}

\maketitle
\begin{center}
\begin{tabular}{l r}

% team members
Members: 	& Obadah Al-Smadi\\
			& Betim Hodza\\
			& Elliot Mai\\
			& Benjamin Niccum\\
        	& Nicholas Pratt\\
Instructor: & Trevor Bakker\end{tabular}
\end{center}

\newpage

\setcounter{tocdepth}{2}
\tableofcontents
\newpage

%%% List of figures and tables (optional)
\listoffigures
\listoftables
\newpage

\section{Security Categorization}
 adwasd
\subsection{Reference Standards}
\begin{itemize}
    \item FIPS 199: Security categorization for federal systems.
    \item NIST SP 800-60: Mapping security categories to information types.
\end{itemize}

\subsection{Impact Levels (Confidentiality, Integrity, Availability)}
\begin{center}
    \rowcolors{2}{lightgray}{white}
    \begin{tabular}{|p{3cm}|p{3.5cm}|p{3.5cm}|p{3.5cm}|}
    \hline
    \rowcolor{navyblue!80}
    \color{white}\textbf{Information Types} & 
    \color{white}\textbf{Confidentiality} & 
    \color{white}\textbf{Integrity} & 
    \color{white}\textbf{Availability} \\ \hline
    
    \makecell{Power Supply/\\Battery} & 
    \makecell[l]{L\\ \scriptsize minimal impact due to 
    \\\scriptsize information only of the 
    \\\scriptsize battery itself } & 
    \makecell[l]{L\\ \scriptsize Due to loss of Integrity }& 
    \makecell[l]{H\\ \scriptsize severe impact to the mission 
    \\\scriptsize can no longer operate
    \\\scriptsize drone without power} \\ \hline
    
    \makecell{Rotors /\\ ECU} & 
    \makecell[l]{L
    \\ \scriptsize Due to loss of Confidentiality} & 
    \makecell[l]{L
    \\ \scriptsize Due to loss of Integrity }& 
    \makecell[l]{L
    \\ \scriptsize Due to loss of availability,\\
    \scriptsize severe impact to the mission 
    \\\scriptsize capability} \\ \hline

    \makecell{FILL} & 
    \makecell[l]{L
    \\ \scriptsize Due to loss of Confidentiality} & 
    \makecell[l]{L
    \\ \scriptsize Due to loss of Integrity }& 
    \makecell[l]{L
    \\ \scriptsize Due to loss of availability,\\
    \scriptsize severe impact to the mission 
    \\\scriptsize capability} \\ \hline

    \makecell{FILL} & 
    \makecell[l]{L
    \\ \scriptsize Due to loss of Confidentiality} & 
    \makecell[l]{L
    \\ \scriptsize Due to loss of Integrity }& 
    \makecell[l]{L
    \\ \scriptsize Due to loss of availability,\\
    \scriptsize severe impact to the mission 
    \\\scriptsize capability} \\ \hline

    \makecell{FILL} & 
    \makecell[l]{L
    \\ \scriptsize Due to loss of Confidentiality} & 
    \makecell[l]{L
    \\ \scriptsize Due to loss of Integrity }& 
    \makecell[l]{L
    \\ \scriptsize Due to loss of availability,\\
    \scriptsize severe impact to the mission 
    \\\scriptsize capability} \\ \hline

    General-Information & L & L & L \\ \hline

    \rowcolor{lightgray}
    \multicolumn{3}{|l|}{\textbf{System Categorization}} & 
    \cellcolor{lightgray} \\ \cline{1-3}
    & Moderate & High & High \\ \hline
    \end{tabular}
\end{center}

\subsection{Overall Categorization}

\subsection{Justification}

\newpage

\section{Risk Management \& Compliance}

Alignment with Risk Management Framework (RMF) per NIST SP 800-37. Categorization informs security control selection (NIST SP 800-53). Ongoing assessment and mitigation per FIPS 200 minimum security controls.
\end{document}